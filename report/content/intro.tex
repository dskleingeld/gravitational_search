Many fields these days depend on good optimization techniques. Metaheuristics such as Particle Swarm Optimization and Simulated Annealing are commonly used. There is no best meta heuristic for all problems. In 2009 a new algorithm inspired by the physics of gravity called \ac{gsa} was introduced\cite{GSA}. 

In \ac{gsa} there are multiple particles spread over the parameter space of the optimization problem. They attract each other based on the fitness of the optimization problem with these parameters, that fitness can be seen as their mass. Newtons second law and his law of gravitation is then used to determine how these particles move at the next time step. For \ac{gsa} to work best the law of gravitation is modified and some randomness is introduced into the system.

\ac{gsa} is weak in its local search ability to improve upon this \ac{gabsa} was created. It uses simulated annealing to improve the local search ability \cite{GABSA}. Here I attempt to reproduce some of the results from the \ac{gabsa} paper. However as there is no publicly available source I will first implement \ac{gsa} before extending it with \ac{gabsa}. 

This report discusses my implementation of \ac{gsa} and \ac{gabsa}, shows how my implementations perform compared to the cited papers before discussing the performance and finally concluding if this replicates the results of the papers.
