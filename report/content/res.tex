Here I will present my GSA implementation compared to the original GSA\cite{GSA} paper for the test functions $f_{1_{gsa}}$ and $f_{2_{gsa}}$ presented in that paper. Then I will compare my GABSA implementation to its paper\cite{GABSA} for the functions $f_{1_{gasba}}$, $f_{2_{gabsa}}$ and $f_{3_{gabsa}}$ presented there. And finally I will show my GSA and GABSA implementation compared for $f_{1_{gsa}}$, $f_{2_{gsa}}$ and $f_{_{3_{gabsa}}}$. 
The suffix to a function $f_{i_\text{suffix}}$ refers to the paper not the method used. In practise the only difference for $f_1$ and $f_2$ is the dimensionallity. See \cref{tab:functions} for the definition of these functions and \cref{tab:params} for the paramaters used. For each function 100 searches where performed.
%
\ctable[
	caption = Test functions used,
	label = tab:functions,
	pos = h,
]{llcc}{}
{
\FL
			 & Function & Range & Optimal value \ML
$f_{1_{gsa}}$& $\sum_{i=1}^d x^2_i$    & $[-100,100]^d$  & $[0,0]^d = 0$ \NN
$f_{2_{gsa}}$& $\sum_{i=1}^d |x_i| + \Pi^d_{i=1} |x_i|$    & $[-10,10]^d$  & $[0,0]^n = 0$ \NN
$f_{3_{gabsa}}$& $0.5+\frac{\text{sin}^2{\sqrt{x_1^2+x_2^2}}-0.5}{\left[1+0.001\left(x_1^2+x_2^2\right)\right]^2}$    & $(-100,100)$  & $(0,0) = 0$ \LL
}

\ctable[
	caption = Paramaters used for the gsa and gasba runs,
	label = tab:params,
	pos = h,
]{lcccccc}{}
{
\FL
		   & Population (N) & Dimension (D) & \# Iterations & $g_0$ & $\alpha$ & $t_0$ \ML
     $f_{gsa}$   & 50             & 30        & 1000          & 100   & 20       & -  \NN
     $f_{gasba}$ & 50             & 2         & 1000          & 100   & 20       & 20 \LL
}
%
\ctable[
	caption = All results,
	label = tab:res,
	pos = h,
]{lcrc}{}
{
\FL
 Function        & Annealing & final fitness                  & succesful search (\%)\ML
 $f_{1_{gsa}}$   & no        & $[5.7 \pm 7.9] \cdot 10^{-19}$ & 100 \NN
                 & yes       & $[6.6 \pm 7.8] \cdot 10^{-19}$ & 100 \NN
 \addlinespace
 $f_{2_{gsa}}$   & no        & $[2.8 \pm 1.5] \cdot 10^{+01}$ & 100 \NN
                 & yes       & $[8.5 \pm 3.0] \cdot 10^{+01}$ & 100 \NN
 \addlinespace
 $f_{3_{gabsa}}$ & no        & $[8.5 \pm 5.5] \cdot 10^{-02}$ & 42  \NN
                 & yes       & $[2.4 \pm 1.2] \cdot 10^{-01}$ & 69  \NN
\addlinespace
 $f_{1_{gabsa}}$ & yes       & $[1.9 \pm 9.7] \cdot 10^{-47}$ & 100 \NN
 $f_{2_{gabsa}}$ & yes       & $[2.1 \pm 19] \cdot 10^{-01}$  & 100 \LL
}
% 5.7E-19 +- 7.9E-19, n: 100 gsa/f1_gsa", 
% 2.8E+01 +- 1.5E+01, n: 100 gsa/f2_gsa", 
% 8.5E-02 +- 5.5E-02, n: 42  gsa/f3_gabsa"
% 1.9E-47 +- 9.7E-47, n: 100 gabsa/f1_gabs             
% 2.1E-01 +- 1.9E+00, n: 100 gabsa/f2_gabs
% 2.4E-01 +- 1.2E-01, n: 69  gabsa/f3_gabs
% 6.6E-19 +- 7.8E-19, n: 100 gabsa/f1_gsa"
% 8.5E+01 +- 3.0E+01, n: 100 gabsa/f2_gsa"            
%
\subsection{The GSA implementation}
In \cref{fig:f1gsa,fig:f2gsa} we see the comparision between my implementation and that of the GSA paper. Then in 
%
\subsection{The GABSA implementation}
%
\subsection{GSA vs GABSA}
%
% \begin{figure}
% 	\centering
% 	\subcaptionbox{This GSA implementation}
% 		[.49\linewidth]{\includegraphics{f1}}
% 	\subcaptionbox{The original GSA implementation}
% 		[.49\linewidth]{\includegraphics{f1}}
% 	\caption{Problem $f_1$ from the GSA paper\cite{GSA} compared}
%     \label{fig:f1gabsa}
% \end{figure}

% \begin{figure}
% 	\centering
% 	\subcaptionbox{This GSA implementation}
% 		[.49\linewidth]{\includegraphics{f1}}
% 	\subcaptionbox{The original GSA implementation}
% 		[.49\linewidth]{\includegraphics{f1}}
% 	\caption{Problem $f_1$ from the GSA paper\cite{GSA} compared}
%     \label{fig:f2gabsa}
% \end{figure}

% \begin{figure}
% 	\centering
% 	\subcaptionbox{This GSA implementation}
% 		[.49\linewidth]{\includegraphics{f1}}
% 	\subcaptionbox{The original GSA implementation}
% 		[.49\linewidth]{\includegraphics{f1}}
% 	\caption{Problem $f_1$ from the GSA paper\cite{GSA} compared}
%     \label{fig:f1gabsa}
% \end{figure}

% \begin{figure}
% 	\centering
% 	\subcaptionbox{This GSA implementation}
% 		[.49\linewidth]{\includegraphics{f1}}
% 	\subcaptionbox{The original GSA implementation}
% 		[.49\linewidth]{\includegraphics{f1}}
% 	\caption{Problem $f_1$ from the GSA paper\cite{GSA} compared}
%     \label{fig:f2gabsa}
% \end{figure}
