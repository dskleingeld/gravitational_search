Here I will present my \ac{gsa} implementation compared to test functions from both papers: $f_{1_{gsa}}$ and $f_{2_{gsa}}$ from the \ac{gsa} paper and by $f_{1_{gasba}}$ $f_{2_{gabsa}}$ and $f_{3_{gabsa}}$ from the \ac{gabsa} paper. Finally I will show my \ac{gsa} and \ac{gabsa} implementation compared for $f_{1_{gsa}}$, $f_{2_{gsa}}$ and $f_{_{3_{gabsa}}}$\footnote{The suffix to a function $f_{i_\text{suffix}}$ refers to the paper not the method used}.

In practise the only difference for $f_1$ and $f_2$ is the dimensionallity. See \cref{tab:functions} for the definition of these functions and \cref{tab:params} for the paramaters used. For each function 100 searches where performed.
%
\ctable[
	caption = Test functions used,
	label = tab:functions,
	pos = h,
]{llcc}{}
{
\FL
			 & Function & Range & Optimal value \ML
$f_{1_{gsa}}$& $\sum_{i=1}^d x^2_i$    & $[-100,100]^d$  & $[0,0]^d = 0$ \NN
$f_{2_{gsa}}$& $\sum_{i=1}^d |x_i| + \Pi^d_{i=1} |x_i|$    & $[-10,10]^d$  & $[0,0]^n = 0$ \NN
$f_{3_{gabsa}}$& $0.5+\frac{\text{sin}^2{\sqrt{x_1^2+x_2^2}}-0.5}{\left[1+0.001\left(x_1^2+x_2^2\right)\right]^2}$    & $(-100,100)$  & $(0,0) = 0$ \LL
}

\ctable[
    caption = {Paramaters used for the gsa and gasba runs},
	label = tab:params,
	pos = h,
]{lcccccc}{}
{
\FL
	Function	   & Population (N) & Dimension (D) & \# Iterations & $g_0$ & $\alpha$ & $t_0$ \ML
     $f_{gsa}$   & 50             & 30        & 1000          & 100   & 20       & -  \NN
     \addlinespace
     $f_{i_{gasba}}$ & 50             & 2         & 50          & 100   & 20       & 20 \NN
     $f_{3_{gasba}}$ & 50             & 2         & 100          & 100   & 20       & 20 \LL
}
%
\ctable[
	caption = Numerical results found here compared to those in the \ac{gsa} and \ac{gabsa} papers,
	label = tab:res,
	pos = h,
]{lcllll}{}
{
\FL
Function        & Annealing & \multicolumn{2}{c}{Average best so far} & \multicolumn{2}{c}{Optimal Solution}\ML
                &     & Here                           & Paper\cite{GSA}      & Here                 & Paper\cite{GABSA}    \ML
$f_{1_{gsa}}$   & no  & $[5.7 \pm 7.9] \cdot 10^{-19}$ & $7.3 \cdot 10^{-11}$ & $2.0 \cdot 10^{-20}$ & -                    \NN
                & yes & $[6.6 \pm 7.8] \cdot 10^{-19}$ & -                    & $5.4 \cdot 10^{-20}$ & -                    \NN
\addlinespace
$f_{2_{gsa}}$   & no  & $[2.8 \pm 1.5] \cdot 10^{+01}$ & $4.0 \cdot 10^{-5}$  & $6.4 $               & -                    \NN
                & yes & $[8.5 \pm 3.0] \cdot 10^{+01}$ & -                    & $28  $               & -                    \NN
\addlinespace
$f_{3_{gabsa}}$ & no  & $[2.8 \pm 2.7] \cdot 10^{-02}$ & -                    & $5.6 \cdot 10^{-17}$ & -                    \NN
                & yes & $[1.8 \pm 1.0] \cdot 10^{-01}$ & -                    & $5.6 \cdot 10^{-17}$ & $2.6 \cdot 10^{-6}$  \NN
\addlinespace
$f_{1_{gabsa}}$ & yes & $[3.9 \pm 5.9] \cdot 10^{-2}$  & -                    & $1.9 \cdot 10^{-7}$  & $1.6 \cdot 10^{-6}$  \NN
$f_{2_{gabsa}}$ & yes & $[2.5 \pm 1.4] \cdot 10^{3}$   & -                    & $1.7 \cdot 10^{-2}$  & $1.2 \cdot 10^{-4}$  \LL
}
%5.7E-19 +- 7.9E-19, best: 2.0E-20  gsa/f1_gsa",
%6.6E-19 +- 7.8E-19, best: 5.4E-20  gabsa/f1_gsa"
 
%2.8E+01 +- 1.5E+01, best: 6.4E+00  gsa/f2_gsa", 
%8.5E+01 +- 3.0E+01, best: 2.8E+01  gabsa/f2_gsa"
 
%2.8E-02 +- 2.7E-02, best: 5.6E-17  gsa/f3_gabsa"
%1.8E-01 +- 1.0E-01, best: 5.6E-17  gabsa/f3_gabs
 
%3.9E-02 +- 5.9E-02, best: 1.9E-07  gabsa/f1_gabs
%2.5E+02 +- 1.4E+03, best: 1.7E-02  gabsa/f2_gabs
%
\subsubsection*{The GSA implementation}
In \cref{fig:f1} we see the comparision between my implementation and that of the \ac{gsa} paper. Note the slight hump at the start enlarged in the top right and the steeper decent.
%
\begin{figure}
	\centering
	\subcaptionbox{My implementation, in blue 100 individual runs in red the average of these runs}
		[.49\linewidth]{\includegraphics{gsa/f1_gsa}}
	\subcaptionbox{Implementation in \ac{gsa} paper, in black the performance of \ac{gsa}}
		[.49\linewidth]{\includegraphics{gsa_paper/f1}}
	\caption{Average best-so-far fitness for function $f_{1_{gsa}}$ compared.}
    \label{fig:f1}
\end{figure}
%
\subsubsection*{The GABSA implementation}
In \cref{fig:f1gabsa,fig:f3gabsa} the comperision between my implementation and that in the \ac{gabsa} paper. Note in \cref{fig:myf1} again the bump at the start and the conversion to discrete values. In \cref{fig:myf2} we see only one out of a hundred runs that follows the \textit{Optional solution} figure for $f_{2_{gabsa}}$ from \cref{tab:res}. The plots from the \ac{gabsa} paper (\cref{fig:bad1,fig:bad2}) are not usable for comparison as they have a linear axis and little detail visible.
%
\begin{figure}
	\centering
	\subcaptionbox{My \ac{gabsa} implementation \label{fig:myf1}}
		[.49\linewidth]{\includegraphics{gabsa/f1_gabsa}}
	\subcaptionbox{The original \ac{gabsa} implementation \label{fig:bad1}}
		[.49\linewidth]{\includegraphics{gabsa_paper/f1}}
	\caption{Problem $f_{2_{gbsa}}$ from the \ac{gabsa} paper\cite{GBSA} compared}
    \label{fig:f1gabsa}
\end{figure}
%
\begin{figure}
	\centering
	\subcaptionbox{My \ac{gabsa} implementation}
		[.49\linewidth]{\includegraphics{gabsa/f3_gabsa}}
	\subcaptionbox{The original \ac{gabsa} implementation \label{fig:bad2}}
		[.49\linewidth]{\includegraphics{gabsa_paper/f3}}
	\caption{Problem $f_{3_{gbsa}}$ from the \ac{gabsa} paper\cite{GBSA} compared}
    \label{fig:f3gabsa}
\end{figure}
%
\subsubsection*{GSA vs GABSA}
A full comparison between \ac{gsa} with and without annealing is out of the scope of this report. In \cref{fig:f2comp} we compare my implementation of \ac{gsa} and \ac{gabsa}.
%
\begin{figure}
	\centering
	\subcaptionbox{\ac{gsa}}
		[.49\linewidth]{\includegraphics{gsa/f2_gsa}}
	\subcaptionbox{GABSA}
		[.49\linewidth]{\includegraphics{gabsa/f2_gsa}}
	\caption{Problem $f_{2_{gsa}}$ from the \ac{gsa} paper\cite{GSA} using my \ac{gsa} and \ac{gabsa} implementation}
    \label{fig:f2comp}
\end{figure}
