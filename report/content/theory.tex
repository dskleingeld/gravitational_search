Here I will concise explanation of both search algorithms starting with GSA as the other is a slight modification upon this. For a more detailed description refer to the GSA\cite{GSA} and GABSA\cite{GABSA} papers.
%
\subsection{GSA} \label{sec:GSA}
The interaction between the particles are governed by Newtons laws, given the mass (fitness\footnote{normalized between $0$ and $1$}) and position (parameters) of each particle (solution) we can derive how they move. Below the $m$ is the mass of our particle for whom we want to know its next position, $M_i$ and $R_i$ the mass and distance to one of the other particles, $G$ a constant in physics here it is a decreasing function, $\vec{a}$ how our particle will accelerate finally $k$ starts out as the total number of particles and liniarly descreasing to one during the seach.
GSA uses a modification of Newtons law of gravitation where $R^2$ is replaced with $R$.
%
\begin{subequations}
\begin{align}
	\vec{F}&=m\vec{a} &            \vec{F}&=\sum_{i=0}^{i=k} \frac{GmM_i}{R_i} \\
	m\vec{a} &= \sum_{i=0}^{i=k} \frac{GmM_i}{R_i} \\
	\vec{a} &= \sum_{i=0}^{i=k} \frac{GM_i}{R_i}
\end{align}
\end{subequations}
%
Note $\vec{a}$ does not depend on $m$, without this result any solution with a fitness of zero would ruin the search turning all values into \texttt{Nan}. To find the paramaters at the next iteration of the search we update its velocity and then its position. The velocity change is randomized (this improves GSAs exploration).
%
\begin{subequations}
\begin{align} \label{eq:paramater_update}
	\vec{v}_{t_2} &= \vec{v}_{t_1} + r \vec{a}\Delta_t \\
	\vec{s}_{t_2} &= \vec{v}_{t_2}\Delta_t & \Delta_t &= 1 \label{eq:pos_update}
\end{align}
\end{subequations}
%
When each particle ends up a the same location the search result converged and we can not search further as the particles will no longer accelerate.
%
\subsection{GABSA}
Adding annealing to the paramater update in \cref{eq:paramater_update} gives us GABSA. To do so we add the condition below after \cref{eq:pos_update}
%
\begin{subequations}[resume]
\begin{align}
	m_{t_1} &= f(\vec{s}_{t_1}) \\
	m_{t_2} &= f(\vec{s}_{t_2}) \\
	exp &= e^{-\frac{m_{t_2}-m_{t_1}}{T}} \\
	\\
	\vec{s}_{t_2} &= \left\{
		\begin{array}{ll}
			\vec{s}_{t_2} & \quad m_{t_2} > m_{t_1} \lor r \leq exp\\ 
			\vec{s}_{t_1} & \quad m_{t_2} \leq m_{t_1}
		\end{array}
	\right.
\end{align}
\end{subequations}
%
Here $T$, the temperature, is a decreasing \textit{cooling} function and $r$ a random number between 0 and 1. As the temperature decreases the chance lowers that we accept new parameters when they lead to a worse solution.
